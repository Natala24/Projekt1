'\documentclass[a4paper,12pt]{article}
\usepackage[utf8]{inputenc}
\usepackage[T1]{fontenc}
\usepackage[polish]{babel}
\usepackage{amsmath}
\usepackage{amsfonts}
\usepackage{amssymb}
\usepackage{hyperref}

\title{PROJEKT 1- Transformacja współrzędnych}
\author{Julia Nowak; Natalia Oknińska}
\date{13 maja 2024}

\begin{document}

\maketitle

\tableofcontents % Dodanie spisu treści

\section{Cel projektu}

Zadanie polegało na utworzeniu skryptu implementującego transformacje:

\begin{itemize}
  \item XYZ -> BLH
  \item BLH -> XYZ
  \item XYZ -> NEUp
  \item BL -> Pl2000
  \item BL -> Pl1992
\end{itemize}

Program miał korzystać z biblioteki argparse do przetwarzania argumentów, z wyłączeniem  odczytu danych z pliku tekstowego, miał także za zadanie tworzyć plik wynikowy.\\ Skrypt był zabezpieczony odpowiednimi klauzulami w przypadku niewłaściwego użytkowania. Dodatkowo, program był udokumentowany i przechowywany repozytorium GitHub.

\section{Wykorzystane narzędzia}

\begin{itemize}
  \item SYSTEMY OPERACYJNE: Microsoft Windows 11 oraz macOS 12.5.1
  \item APLIKACJE: Spyder; Python 3.9.13
 \end{itemize}

\section{Opis przebiegu ćwiczenia}

Pierwszym krokiem, który podjęliśmy, było zdefiniowanie klasy o nazwie "Transformations". Ta klasa ściśle współpracuje z biblioteką `argparse`, która jest standardową biblioteką w Pythonie, więc nie wymaga dodatkowej instalacji przez użytkownika. Wykorzystując algorytmy napisane wcześniej w trzecim semestrze, zaimplementowaliśmy różne transformacje w ramach tej klasy, zgodnie z zadeklarowaną specyfikacją, korzystając przy tym z bibliotek takich jak `math` czy `NumPy`, co znacznie ułatwiło nam pisanie kodu, zwiększyło jego czytelność i estetykę.

Następnym etapem było dodanie klauzuli if name == " main ", która wykorzystuje wcześniej wspomnianą bibliotekę `argparse`. Dzięki tej bibliotece użytkownik może przekazywać argumenty poprzez wiersz poleceń. Stworzyliśmy trzy zdefiniowane argumenty, jakie użytkownik może przekazać: plik (z którego użytkownik chce przeliczyć dane), elipsoidę oraz rodzaj transformacji do wykonania. Dla ostatnich dwóch argumentów utworzyliśmy słowniki, z których program korzysta, gdy użytkownik podaje ich wartości, co stanowi dodatkowe zabezpieczenie przed sytuacją, gdy użytkownik poda elipsoidę lub transformację, które nie są obsługiwane przez program. W przypadku błędnych argumentów użyliśmy klauzuli `try-except`, aby program mógł obsłużyć wyjątek i wyświetlić odpowiedni komunikat dla użytkownika.

\section{Rezultat}
Poniżej znajduje się link do naszego projektu: \\
https://github.com/Natala24/Projekt1

\section{Nabyte umejętności}

\begin{itemize}
\item Kodowanie w paradygmacie obiektowym w Pythonie
\item Implementacja algorytmów z zewnętrznych źródeł
\item Tworzenie dokumentów w LaTeX
\item Współpraca zespołowa przy użyciu systemu kontroli wersji Git
\item Tworzenie narzędzi w interfejsie tekstowym z możliwością przyjmowania   
\item Tworzenie i redagowanie tworzeprzydatnej dokumentacji
 \end{itemize}

\section{Spostrzeżnia i trudności}

Pierwszą napotkaną trudnością było zainsalowanie potrzebych aplikikacji do stworzenia projeku dla systemu operacyjnego macOS - w szczegolności instalacja programu Git. Po przeczytaniu kilku artykułów ze wskazowkami oraz po odbytych konsultacjach program został zainstalowy w prawidłowy sposoób. W momencie w którym  nasze konta zostały poprawie działać praca nabrała tempa. \\

Dodatkowym wyzwaniems było również napisanie funkcji do rozpakowywania pliku. Trudno jest tworzyć funkcję do rozpakowywania pliku, którego struktury nie widać ani nie jest jasne - chodzi tutaj o to, że tworzymy uniwersalną funkcję dla użytkownika, a nie taką, która byłaby zostosowana do konkretnego pliku.\\

Głównym spostrzeżeniem po skończeniu projektu był pozytywny wpływ na organizajcę naszej pracy zespołowej. Ten projekt pokazał nam jak ważna jest dobra organizacja i podział obowiązków. 

\section{Wnioski}

Projekt nauczył nas korzystania z wielu przydatnych narzędzi do współpracy. Była to możliwość przećwiczenia wszystkich zagadnień omawianych na ćwiczeniach z informatyki geodezyjnej II. 

\section{Bibliografia}
\begin{thebibliography}{9}
\bibitem{reference1} M.Borkowski, B.Przybylski, \emph{Ksiązka kucharska Latex}, 2015
\bibitem{reference1} M.Doob, \emph{Łagodne wprowadzenie do Tex-a}, 2002

\end{thebibliography}

\end{document}